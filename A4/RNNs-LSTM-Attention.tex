\documentclass[12pt]{article}
\usepackage{lingmacros}
\usepackage{tree-dvips}
% hyper links
\usepackage{hyperref}
\usepackage[utf8]{inputenc}
\usepackage{amsmath}
\usepackage{amsfonts}
\usepackage{bbm}
% Formatting quotes properly
\usepackage[english]{babel}
\usepackage[autostyle, english = american]{csquotes}
\MakeOuterQuote{"}


\begin{document}
\noindent Author: Benjamin Smidt

\noindent Created: October 18th, 2022

\noindent Last Updated: October 23rd, 2022
\begin{center}
\section*{Assignment 4: RNNs, LSTM, and Attention with Image Captioning}
\end{center}

\paragraph{} \emph{Note to reader.} 

This is my work for assignment four (A4) of Michigan's course
\href{https://web.eecs.umich.edu/~justincj/teaching/eecs498/WI2022/}
{EECS 498: Deep Learning for Computer Vision}. The majority of explanations and understanding are 
derived from \href{https://www.youtube.com/watch?v=dJYGatp4SvA&list=PL5-TkQAfAZFbzxjBHtzdVCWE0Zbhomg7r&index=1}
{Justin Johnson's Lectures} and \href{http://cs231n.stanford.edu/schedule.html}{Stanford's CS 231N Lecture Notes}.
This document is meant to be used as a reference, 
explanation, and resource for the assignment, not necessarily a comprehensive overview
of Neural Networks. If there's a typo or a correction needs to be made, feel free to 
email me at benjamin.smidt@utexas.edu so I can fix it. Thank you! I hope you find this 
document helpful.

\tableofcontents{}

\newpage

\section{Vanilla Recurrent Neural Networks}

\subsection{RNN Forward}
Recurrent neural networks are very powerful in their ability to process and output 
variable length data. Said another way, RNNs can be fed different length inputs as 
well predict different length outputs making them a powerful and useful paradigm in 
many applications. We'll go into more depth as we go along but for now we'll start 
with vanilla recurrent neural networks. 
\begin{equation*}
    h_t = f_w (h_{t-1}, x_t)
\end{equation*}
To achieve variable length inputs and outputs we need to change our neural network 
model a bit. Instead of having some predefined network size, we have a function that
takes two inputs: the output of the previous computation (also known as the "hidden state")
and some data input (usually interpreted as a sequence). 
See \href{https://www.google.com/search?q=recurrent+neural+network&sxsrf=ALiCzsaznqUkAxJ_FZnLauL7_6Z3AD132g:1666096199658&source=lnms&tbm=isch&sa=X&ved=2ahUKEwizlpGB5On6AhVtmWoFHfgSCc0Q_AUoAXoECAIQAw&biw=1496&bih=1138&dpr=1.13#imgrc=iC7Ot7uyj4lzoM}
{this picture} for a visual. For vanilla neural networks (also known as "Elman RNNs") 
we use the following function. 
\begin{equation}
    h_t = tanh(h_{t-1}W_{hh} + x_tW_{xh} + b)
\end{equation}
where $h_t$ is the current state, $x_t$ is the input data, $W_{hh}$ is our (reused) 
weight matrix for the hidden state input, and $W_{xh}$ is our (reused) weight 
matrix for the current input $x_t$. We also add a bias $b$. To be clear, $W_{hh}$ and 
$W_{xh}$ do not change at all beween time steps. They are the same set of parameters 
throughout the neural network's computation. For out first function, \emph{rnn-forward}, 
we simply write down Eq. (1) in code and store our needed variables in \emph{cache} 
for backpropagation. 

If you're wondering about initialization and how to know when to stop computing $h_t$, 
keep reading. I'll answer those and other questions as we go along. 

\subsection{RNN Backward}
Let's look at backpropagating a given time step given our function. Recall that 
\begin{equation*}
    tanh(z) = \frac{e^z - e^{-z}}{e^z + e^{-z}}
\end{equation*}
Thus, by quotient rule, our derivative is as follows
\begin{equation*}
    \frac{\partial \; tanh(z)}{\partial z} = 
    \frac{(e^z + e^{-z})(e^z + e^{-z}) 
    - (e^z - e^{-z})(e^z - e^{-z})}
    {(e^z + e^{-z})^2}
\end{equation*} \begin{equation*}
    \frac{\partial \; tanh(z)}{\partial z} = 
    \frac{(e^z + e^{-z})^2 - (e^z - e^{-z})^2}
    {(e^z + e^{-z})^2}
\end{equation*} \begin{equation*}
    \frac{\partial \; tanh(z)}{\partial z} = 
    \frac{(e^z + e^{-z})^2}
    {(e^z + e^{-z})^2} -
    \frac{(e^z - e^{-z})^2}
    {(e^z + e^{-z})^2}
\end{equation*}\begin{equation*}
    \frac{\partial \; tanh(z)}{\partial z} = 
    1 - tanh^2(z)
\end{equation*} 
If we set $z = W_{hh}h_{t-1} + W_{xh}x_t + b$, we can get the first term in our backpropagation.
I'll write our original function here as well for clarity. 
\begin{equation}
    h_t = tanh(z)
\end{equation}
\begin{equation*}
    \frac{\partial loss}{\partial z} = 
    \frac{\partial loss\; z}{\partial h_t} \odot
    (1 - tanh^2(z))
\end{equation*} 
Where $\frac{\partial loss\; z}{\partial h_t}$ is passed down to us from some function upstream 
and $\odot$ indicates elementwise multiplication (work out the shapes!). 
Moving forward (or backward I guess) in our backpropagation, we'll next work 
on each of the variables inside the $tanh$ function starting with $W_{hh}$ and 
$W_{xh}$. 
\begin{equation*}
    \frac{\partial \; z}{\partial W_{hh}} = h_{t-1} \; \; \; \text{and} \; \; \;
    \frac{\partial \; z}{\partial W_{xh}} = x_t
\end{equation*}
Thus 
\begin{equation*}
    \frac{\partial \; h_t}{\partial z} \frac{\partial \; z}{\partial W_{hh}} = 
    h_{t-1}^T \; [1 - tanh^2(z)] 
\end{equation*}
\begin{equation*}
    \frac{\partial \; h_t}{\partial z} \frac{\partial \; z}{\partial W_{xh}} = 
    x_t^T [1 - tanh^2(z)]
\end{equation*}
where $z = W_{hh}h_{t-1} + W_{xh}x_t + b$ and the transposes are derived by the shape 
convention. Next we have $h_{t-1}$ and $x_t$. 
\begin{equation*}
    \frac{\partial \; z}{\partial h_{t-1}} = W_{hh} \; \; \; \text{and} \; \; \;
    \frac{\partial \; z}{\partial x_t} = W_{xh}
\end{equation*}
Thus 
\begin{equation*}
    \frac{\partial \; h_t}{\partial z} \frac{\partial \; z}{\partial h_{t-1}} = 
    [1 - tanh^2(z)] \; W_{hh}^T 
\end{equation*}
\begin{equation*}
    \frac{\partial \; h_t}{\partial z} \frac{\partial \; z}{\partial x_t} = 
    [1 - tanh^2(z)] \; W_{xh}^T 
\end{equation*}
And finally for our bias 
\begin{equation*}
    \frac{\partial \; z}{\partial b} = 1
\end{equation*} \begin{equation*}
    \frac{\partial \; h_t}{\partial z} \frac{\partial \; z}{\partial b} = 
    [1 - tanh^2(z)] 
\end{equation*}
Of course, we'll have to manipulate the bias $b$ more when we program the backpropagation 
since $b$ is actually broadcast over the the outputs which needs to be accounted for 
in our backpropagation (we sum over the rows, which is dimension 0). 

\subsection{RNN Forward}
This is where some of our initial questions need some answers. 
~\\
~\\
How do we initialize the network? 
(i.e. where does the first $h_{t-1}$ come from?). We simply initialize it as a separate matrix 
and make it a learnable parameter of the network. 
~\\
~\\
When do we stop our recursive calls to $f_w$? When the input sequence has run out. For instance, 
given $T$ time steps for a set of data, we recursively compute $f_w$ until we get to the 
last time step where we throw in the towel and compute our loss function. 
~\\
~\\
Speaking of loss functions, how is it computed? Well, because recurrent neural networks 
are very malleable, how you compute the loss depends on the type of recurrent 
neural network you use. I really like the visuals shown on \href{https://calvinfeng.gitbook.io/machine-learning-notebook/supervised-learning/recurrent-neural-network/recurrent_neural_networks}
{this website}. Our network is a one to many relationship, meaning we have a loss computed 
for each hidden state $h_i$. Thus, we must compute the gradient with respect to the loss 
produced by the current state $h_i$ as well as with respect to all the downstream 
states $h_d > h_i$. This sounds more difficult than it really is in practice. 
~\\
~\\
This forward function isn't really anything new to use so I'll leave you to just read the code in
the notebook. The only detail to note is that I chose to transpose the input $x$ (shape $N$ x $T$ x $D$)
for computing the forward pass to make the computation more clear. This led me to have to take 
the transpose of the output $h$ (tensor of hidden states with shape $N$ x $T$ x $H$). 

\subsection{RNN Backward}
As I briefly mentioned above, computing the gradient for any given time step is
a little more complicated than what we're used to. It's really not much different
though if you're used to computing gradients already. First, see 
\href{https://calvinfeng.gitbook.io/machine-learning-notebook/supervised-learning/recurrent-neural-network/recurrent_neural_networks}
{this website} for a picture of what a \emph{one-to-many} relationship looks like 
for recurrent neural networks. 
~\\
~\\
In a one-to-many RNN, a loss function is computed for each time step. Thus, the loss 
computed for a given time step depends on all the time steps before it. This means, 
for a given time step, our gradient depends on the loss computed at that time step 
as well as the loss computed for all the time steps after it. 
~\\
~\\
In practice this isn't too difficult. The time step $h_{t + 1}$ will pass back some 
gradient to time step $h_t$. Since this gradient is passed back from every time 
step ahead of $h_t$, it embeds all the gradient with respect to all the loss functions 
after $h_t$ (so excluding the loss function calculated on time step $h_t$. Thus, 
we simply add the gradient passed back from $h_{t+1}$ and the gradient with respect 
to the loss function computed at $h_t$ and pass this value to our 
\emph{rnn-step-backward} function. And that's it! 
~\\
~\\
Again, the details here are pretty simply once you understand the high level 
concept so I won't explain the code. We're just backpropagating like normal 
with a couple extra steps added in between. 

\section{RNN for Image Captioning}

Initially we implement some functions that we need to do our image captioning. Namely, 
we look at MobileNet v2 architecture, create vectors for our word embeddings, and 
look at a temporal affine layer. I'm not going go over these because they're pretty 
straightforward and we can just use PyTorch's API's for a lot of it to make things 
simple. 

\subsection{RNN Captioning Forward}

\subsubsection{Overview}
This architecture was a little confusing to me at first I'm not going to lie. However, 
it makes perfect sense once I went through it step by step. Let's do that. 
~\\
~\\
Remember that we're given an image and associated captions for a given number of examples. 
Our job is to use this data to train a network such that, if we only fed it an input 
picture, it would give us a caption. To do this we're using a many-to-many RNN architecture. 
Our initial state is a feature vector extracted from our image, 
and each time step is fed the vector representation of the previous word. Then, at each time 
step, it predicts the next word which we compute softmax loss on and use for backpropagation. 

\subsubsection{Input Image Feature Vector}
There are quite a few operational details here but that's the big picture. Let's start with extracting 
our feature vector from our input image. We'll use the \emph{FeatureExtractor} class
(implemented for us) to do this. What this class does is use the MobileNet v2 model 
to extract the features for us (as opposed to training a whole neural network for this 
purpose only). 
~\\
~\\
We'll adjust the network a little bit by chopping off the last two layers 
(FC-1000 and softmax) leaving us with a feature tensor of 1280 x 4 x 4 as the output. We'll then use 
average pooling on dimensions 1 and 2 to get our feature vector of length 1280. Before we 
talk about how we input this into our RNN, let's quickly review our RNN architecture. 
~\\
~\\
Each step of our RNN takes two inputs: the previous hidden state $h_t$ and the current 
input $x_t$. We use $h_{t+1} = tanh(W_{hh}h_{t} + W_{xh}x + b)$ where $W_{hh}$ and $W_{xh}$ 
are the same matrix for all time steps. The output, $h_{t+1}$, is then fed into the next 
step along with the next input $x_t+1$. In our case, each $x$ is a vector representation of 
a word. 
~\\
~\\
Staying on topic though let's look at some shapes. $x$ will be an $N$ x $W$ dimensional tensor 
where $N$ is the number of examples in our current batch and $W$ is the length of our vector 
representations of a given word (so we have $N$ words represented each as vectors of length $W$). 
A given $h_t$ will be of shape $N$ x $H$ where $H$ is the vector length we choose to represent 
a given state for a given example. 
~\\
~\\
Okay, so back to our input feature vector of length 1280. Remember, because we always do training 
in batches, we'll have $N$ input feature vectors of length 1280. Thus, if we tried to use 
this as our current input as $h_0$ it wouldn't work because we're trying to input a tensor of 
shape $N$ x 1280 which isn't the same shape as our general hidden state $h_t$ of $N$ x $H$ (I mean, guess it could be 
but it isn't for our implementation). 
~\\
~\\
Thus, we use an affine (linear) transformation (matrix 
multiplication) to convert it to the shape we want. In particulary, we multiply our feature 
tensor of shape $N$ x 1280 by our affine transformation tensor (matrix) of shape 1280 x $H$ 
to get our initial state $h_0$ of shape $N$ x $H$. This affine transformation matrix is a 
learnable part of our network meaning we will perform backprop on it (although we let PyTorch 
do that for us this assignment). 

\subsubsection{RNN, Loss, and Backprop}
I'm not going to go super in depth on the details since I already explained them 
above when implementing the loss and backpropagation for RNNs, but there are some 
things I need to clear up. First, I've stated before, each time step is associated with a 
predicted next word given the previous hidden state and the previous word (represented 
by a vector). How does this work exactly though? 
~\\
~\\
Well, we again use an affine transformation to turn a given example's hidden vector of 
length $H$ into a hidden vector of length $N$. That is, we matrix multiply the vector by 
a matrix of shape $H$ x $V$ where $V$ is the length of our vocabulary (more on this in a 
second). Thus, a given time step $h_t$ with $N$ hidden vectors can be matrix multiplied 
to produce a \emph{score} tensor of shape $N$ x $V$ ($N$ x $H$ @ $H$ x $V$). 
~\\
~\\
Next, we can compute our softmax loss function on each example (so the $V$ dimension, 
dimension 1 in this case) using the ground truth labels of what the next word should 
actually be. This is done for every time step in our RNN. Thus, our backpropagation needs
to include the loss function at the current time step and all the time steps after it 
for a given time step $h_t$ (as discussed in the previous section, the idea is the exact 
same). 
~\\
~\\
Also, please note that the affine function used to transform $h_t$ into our scores tensor 
is another learnable parameter of the network (meaning we again do backpropagation on it). 
It is the same function used at every time step though, similar to $W_{hh}$ and $W_{xh}$. 

\subsubsection{Word Embeddings}
That last part, that I've kind of beat around the bush, is our word embeddings. This part
isn't very difficult to understand but I do want to make sure it's clear. Essentially, 
the best method to represent words (so far) is using vectors. In our case, a vector of 
length $W$ represents a single word. I won't get into why this is the case or how we come 
up with those vectors. 
~\\
~\\
For this assignment, the vector representations don't actually mean anything. They're just 
unique identifiers for words. Typically though, word vectors \emph{do} have meaning associated 
with them. The numbers in the vectors actually indicate something about semantic meaning or
relationship to other words. We use vectors to represent words because this is standard 
practice in NLP and we need to do so to make our neural network work ($x$ being a vector
instead of just a scalar gives the network more parameters with which to store information 
and make better predictions). 

\section{LSTM}
Onto LSTMs, which stands for Long Short Term Memory. I don't think this the greatest name 
(seems kind of misleading I guess) but I see where the name came from. In essence, LSTMs 
solve the problem of vanishing and exploding gradients that are pervasive in long 
RNNs. LSTM architecture is \emph{very} prevalent, so get used to seeing it and understand 
it well. What is it? It's a little complicated at first but we'll take step by step. 
~\\
~\\
\subsection{LSTM Forward}
First, in addition to having a hidden state at a given time step $h_t$, we have a cell state 
$c_t$. The hidden state and cell state are the same dimension, in particular $N$ x $H$. Our $x_t$ 
doesn't change, it's still of shape $N$ x $D$. Our weight matrices however, $W_{xh}$ and $W_{hh}$,
are different from previously though. We'll use $D$ x $4H$ for $W_{xh}$ and $H$ x $4H$ for $W_{hh}$. 
This leaves us with the following (very similar from before) equation. 
\begin{equation*}
    a = x_t W_{xh} + h_{t-1} W_{hh}  + b
\end{equation*}
to compute the \emph{activation vector} $a$ of shape $N$ x $4H$ (instead of $h_t$ as we did previously). 
We then split up this activation 
vector into four vectors: $a_i$, $a_f$, $a_o$, and $a_g$ where all four vectors are of shape 
$N$ x $H$. Each of them are pushed through the following nonlinearities: 
\begin{equation*}
    i = \sigma(a_i) \; \; \; f = \sigma(a_f) \; \; \; o = \sigma(a_o) \; \; \; g = tanh(a_g)
\end{equation*}
where $\sigma(x)$ is the sigmoid function. We then use these to compute the next cell state
\begin{equation*}
    c_t = f \odot c_{t-1} + i \odot g 
\end{equation*}
and finally the next hidden state. 
\begin{equation*}
    h_t = o \odot tanh(c_t)
\end{equation*}
Okay, so that's how we compute a forward pass of the LSTM architecture. But why? Well, we 
know it solves the vanishing and exploding gradient problem. But how? Another good question. 
For that, let's talk about LSTM RNN backpropagation. The implementation of the LSTM 
forward pass is pretty cut and dry. It's quite similar to our Elman RNN implementation 
just with a few more gadgets and gizmos. 

\subsection{LSTM Backward}
Ah, the reason the LSTM architecture exists: gradient flow. Before we talk about how LSTMs 
solve gradient flow, let's review why Vanilla RNNs struggle to have good gradient flow for 
long sequences. The \href{https://web.eecs.umich.edu/~justincj/slides/eecs498/498_FA2019_lecture12.pdf}
{EECS 498 Lecture Slides pg. 86} give a nice visual for this. 
~\\
~\\
Essentially, during backpropagation we must propagate through each time step $h_t$. By calculus, 
the gradient will be whatever weight matrix we multiply by $h_t$. In our case, we have 
\begin{equation*}
    h_{t+1} = tanh(h_{t}W_{hh} + x_tW_{xh} + b)
\end{equation*}
Thus, our gradient with respect to $h_t$ at any time step is $W_{hh}$ (with the 
$tanh()$ derivative in their as well). This results in a continual multiplication 
of matrix $W_{hh}$ on itself for every time step in the network to propagate all 
the way to the beginning. As such, values greater than 1 quickly explode and those 
less than 1 quickly vanish, leaving us with a serious gradient flow problem. 
~\\
~\\
So, how do LSTMs fix this issue? Well if you go to \href{https://web.eecs.umich.edu/~justincj/slides/eecs498/498_FA2019_lecture12.pdf}
{EECS 498 Lecture Slides pgs. 94 - 97}, you can visually see the difference in gradient 
flow using LSTMs. We could go through the entire mechanics of backpropagation but you don't 
need to do so to see that LSTMs have many different connections and pathways. The information 
flow has a lot of options including a fair amount of additions (which play especially nice since
they distribute the gradient, can't decrease, and don't explode). 
~\\
~\\
This is essentially why LSTMs improve gradient flow: the allow more connections and don't 
repeatedly rely on a given matrix $W_{hh}$. This is not dissimilar to Residual Networks and 
their use of skip connections to improve gradient flow. In fact, befor ResNet was a thing, 
an LSTM like architecture was proposed to solve the same gradient issues ResNet solved. So, 
you can see they very much share the same intuition. 

\subsection{Attention}

\subsubsection{Intuition}
Attention! Okay I'll try not to make any silly attention jokes. In all honesty, attention is 
one of the most intruiguing, yet difficult (the generalization to tranformers especially) 
concepts we've studied so far in the course. 
~\\
~\\
Let's begin with some intuition. We have some data, a picture in our case, that we want 
to use to predict something, captions in our case. Our output is a sequence and while 
we could use all the data at each time step to make a prediction (for that time step), 
it's reasonable to want to be able to "focus" on something. That is, it's reasonable to 
give more attention (weight) to some parts of our data than others and letting this be 
a learnable parameter of our network. Hopefully, our network can then focus on what is 
most relevant in the data to predicting the next time step and make a better prediction 
because of it. 

\subsubsection{Basic Attention Layer}
This is all fine and dandy, but how do we actually do this? Well, we compute 
alignment scores over our input data (our picture) and normalize them using 
softmax. See \href{https://web.eecs.umich.edu/~justincj/slides/eecs498/498_FA2019_lecture13.pdf}
{EECS 498 Lecture Slides pg. 15} for a picture. The computation of our alignment scores
are usually given by 
\begin{equation*}
    e = f_{att}(q, X)
\end{equation*}
where $q$ (shape $D_Q$) is our \emph{query vector}, often a time step of the encoder (which we don't have in 
this network) or some data (which we do have in this network). In our case, $q$ is the image we're captioning. 
$X$ is a matrix of input vectors (shape $N_x$ x $D_Q$). As a note, $D_Q$ is the same dimension as 
what we've been referring to as $H$ throughout this assignment (often times $q$ is $h_{t-1}$ with shape 
$H$ using our previous notation). 
~\\
~\\
So, what's our $f_{att}$ function? Generally just a dot product. It's simple, computationally
efficient and quite robust. In particular we have 
\begin{equation*}
    e = \frac{q X^T}{\sqrt{D_Q}} \; \; \; e_i = \frac{q \cdot X_i}{\sqrt{D_Q}}
\end{equation*}
where $e$ is a vector of length $N_x$. You may be wondering why we're dividing by $\sqrt{D_Q}$. 
Remember that large values don't play well with softmax, so we normalize by $\sqrt{D_Q}$. Now, 
our vector $e$ is a vector of alignment scores. We apply softmax to it so convert those 
alignment scores to weights upon which use again on our vector $X$ to produce an output vector 
$Y$ that we'll be able to use as an input into a given time step of our network. 

\subsubsection{}



\end{document}