\documentclass[12pt]{article}
\usepackage{lingmacros}
\usepackage{tree-dvips}
% hyper links
\usepackage{hyperref}
\usepackage[utf8]{inputenc}
\usepackage{amsmath}
\usepackage{amsfonts}
\usepackage{bbm}
% Formatting quotes properly
\usepackage[english]{babel}
\usepackage[autostyle, english = american]{csquotes}
\MakeOuterQuote{"}


\begin{document}
\noindent Author: Benjamin Smidt

\noindent Created: October 18th, 2022

\noindent Last Updated: October 18th, 2022
\begin{center}
\section*{Assignment 4: RNNs, LSTM, and Attention with Image Captioning}
\end{center}

\paragraph{} \emph{Note to reader.} 

This is my work for assignment four (A4) of Michigan's course
\href{https://web.eecs.umich.edu/~justincj/teaching/eecs498/WI2022/}
{EECS 498: Deep Learning for Computer Vision}. The majority of explanations and understanding are 
derived from \href{https://www.youtube.com/watch?v=dJYGatp4SvA&list=PL5-TkQAfAZFbzxjBHtzdVCWE0Zbhomg7r&index=1}
{Justin Johnson's Lectures} and \href{http://cs231n.stanford.edu/schedule.html}{Stanford's CS 231N Lecture Notes}.
This document is meant to be used as a reference, 
explanation, and resource for the assignment, not necessarily a comprehensive overview
of Neural Networks. If there's a typo or a correction needs to be made, feel free to 
email me at benjamin.smidt@utexas.edu so I can fix it. Thank you! I hope you find this 
document helpful.

\tableofcontents{}

\newpage

\section{Vanilla Recurrent Neural Networks}

\subsection{RNN Forward}
Recurrent neural networks are very powerful in their ability to process and output 
variable length data. Said another way, RNNs can be fed different length inputs as 
well predict different length outputs making them a powerful and useful paradigm in 
many applications. We'll go into more depth as we go along but for now we'll start 
with vanilla recurrent neural networks. 
\begin{equation*}
    h_t = f_w (h_{t-1}, x_t)
\end{equation*}
To achieve variable length inputs and outputs we need to change our neural network 
model a bit. Instead of having some predefined network size, we have a function that
takes two inputs: the output of the previous computation (also known as the "hidden state")
and some data input (usually interpreted as a sequence). 
See \href{https://www.google.com/search?q=recurrent+neural+network&sxsrf=ALiCzsaznqUkAxJ_FZnLauL7_6Z3AD132g:1666096199658&source=lnms&tbm=isch&sa=X&ved=2ahUKEwizlpGB5On6AhVtmWoFHfgSCc0Q_AUoAXoECAIQAw&biw=1496&bih=1138&dpr=1.13#imgrc=iC7Ot7uyj4lzoM}
{this picture} for a visual. For vanilla neural networks (also known as "Elman RNNs") 
we use the following function. 
\begin{equation}
    h_t = tanh(W_{hh}h_{t-1} + W_{xh}x_t + b)
\end{equation}
where $h_t$ is the current state, $x_t$ is the input data, $W_{hh}$ is our (reused) 
weight matrix for the hidden state input, and $W_{xh}$ is our (reused) weight 
matrix for the current input $x_t$. We also add a bias $b$. To be clear, $W_{hh}$ and 
$W_{xh}$ do not change at all beween time steps. They are the same set of parameters 
throughout the neural network's computation. For out first function, \emph{rnn-forward}, 
we simply write down Eq. (1) in code and store our needed variables in \emph{cache} 
for backpropagation. 

If you're wondering about initialization and how to know when to stop computing $h_t$, 
keep reading. I'll answer those and other questions as we go along. 

\subsection{RNN Backward}
Let's look at backpropagating a given time step given our function. Recall that 
\begin{equation*}
    tanh(z) = \frac{e^z - e^{-z}}{e^z + e^{-z}}
\end{equation*}
Thus, by quotient rule, our derivative is as follows
\begin{equation*}
    \frac{\partial \; tanh(z)}{\partial z} = 
    \frac{(e^z + e^{-z})(e^z + e^{-z}) 
    - (e^z - e^{-z})(e^z - e^{-z})}
    {(e^z + e^{-z})^2}
\end{equation*} \begin{equation*}
    \frac{\partial \; tanh(z)}{\partial z} = 
    \frac{(e^z + e^{-z})^2 - (e^z - e^{-z})^2}
    {(e^z + e^{-z})^2}
\end{equation*} \begin{equation*}
    \frac{\partial \; tanh(z)}{\partial z} = 
    \frac{(e^z + e^{-z})^2}
    {(e^z + e^{-z})^2} -
    \frac{(e^z - e^{-z})^2}
    {(e^z + e^{-z})^2}
\end{equation*}\begin{equation*}
    \frac{\partial \; tanh(z)}{\partial z} = 
    1 - tanh^2(z)
\end{equation*} 
If we set $z = W_{hh}h_{t-1} + W_{xh}x_t + b$, we can get the first term in our backpropagation.
I'll write our original function here as well for clarity. 
\begin{equation}
    h_t = tanh(z)
\end{equation}
\begin{equation*}
    \frac{\partial loss}{\partial z} = 
    \frac{\partial loss\; z}{\partial h_t} \odot
    (1 - tanh^2(z))
\end{equation*} 
Where $\frac{\partial loss\; z}{\partial h_t}$ is passed down to us from some function upstream 
and $\odot$ indicates elementwise multiplication (work out the shapes!). 
Moving forward (or backward I guess) in our backpropagation, we'll next work 
on each of the variables inside the $tanh$ function starting with $W_{hh}$ and 
$W_{xh}$. 
\begin{equation*}
    \frac{\partial \; z}{\partial W_{hh}} = h_{t-1} \; \; \; \text{and} \; \; \;
    \frac{\partial \; z}{\partial W_{xh}} = x_t
\end{equation*}
Thus 
\begin{equation*}
    \frac{\partial \; h_t}{\partial z} \frac{\partial \; z}{\partial W_{hh}} = 
    h_{t-1}^T \; [1 - tanh^2(z)] 
\end{equation*}
\begin{equation*}
    \frac{\partial \; h_t}{\partial z} \frac{\partial \; z}{\partial W_{xh}} = 
    x_t^T [1 - tanh^2(z)]
\end{equation*}
where $z = W_{hh}h_{t-1} + W_{xh}x_t + b$ and the transposes are derived by the shape 
convention. Next we have $h_{t-1}$ and $x_t$. 
\begin{equation*}
    \frac{\partial \; z}{\partial h_{t-1}} = W_{hh} \; \; \; \text{and} \; \; \;
    \frac{\partial \; z}{\partial x_t} = W_{xh}
\end{equation*}
Thus 
\begin{equation*}
    \frac{\partial \; h_t}{\partial z} \frac{\partial \; z}{\partial h_{t-1}} = 
    [1 - tanh^2(z)] \; W_{hh}^T 
\end{equation*}
\begin{equation*}
    \frac{\partial \; h_t}{\partial z} \frac{\partial \; z}{\partial x_t} = 
    [1 - tanh^2(z)] \; W_{xh}^T 
\end{equation*}
And finally for our bias 
\begin{equation*}
    \frac{\partial \; z}{\partial b} = 1
\end{equation*} \begin{equation*}
    \frac{\partial \; h_t}{\partial z} \frac{\partial \; z}{\partial b} = 
    [1 - tanh^2(z)] 
\end{equation*}
Of course, we'll have to manipulate the bias $b$ more when we program the backpropagation 
since $b$ is actually broadcast over the the outputs which needs to be accounted for 
in our backpropagation (we sum over the rows, which is dimension 0). 

\subsection{RNN Forward}
This is where some of our initial questions need some answers. 
~\\
~\\
How do we initialize the network? 
(i.e. where does the first $h_{t-1}$ come from?). We simply initialize it as a separate matrix 
and make it a learnable parameter of the network. 
~\\
~\\
When do we stop our recursive calls to $f_w$? When the input sequence has run out. For instance, 
given $T$ time steps for a set of data, we recursively compute $f_w$ until we get to the 
last time step where we throw in the towel and compute our loss function. 
~\\
~\\
Speaking of loss functions, how is it computed? Well, because recurrent neural networks 
are very malleable, how you compute the loss depends on the type of recurrent 
neural network you use. I really like the visuals shown on \href{https://calvinfeng.gitbook.io/machine-learning-notebook/supervised-learning/recurrent-neural-network/recurrent_neural_networks}
{this website}. Our network is a one to many relationship, meaning we have a loss computed 
for each hidden state $h_i$. Thus, we must compute the gradient with respect to the loss 
produced by the current state $h_i$ as well as with respect to all the downstream 
states $h_d > h_i$. This sounds more difficult than it really is in practice. 
~\\
~\\
This forward function isn't really anything new to use so I'll leave you to just read the code in
the notebook. The only detail to note is that I chose to transpose the input $x$ (shape $N$ x $T$ x $D$)
for computing the forward pass to make the computation more clear. This led me to have to take 
the transpose of the output $h$ (tensor of hidden states with shape $N$ x $T$ x $H$). 

\subsection{RNN Backward}
As I briefly mentioned above, computing the gradient for any given time step is
a little more complicated than what we're used to. It's really not much different
though if you're used to computing gradients already. First, see 
\href{https://calvinfeng.gitbook.io/machine-learning-notebook/supervised-learning/recurrent-neural-network/recurrent_neural_networks}
{this website} for a picture of what a \emph{one-to-many} relationship looks like 
for recurrent neural networks. 
~\\
~\\
In a one-to-many RNN, a loss function is computed for each time step. Thus, the loss 
computed for a given time step depends on all the time steps before it. This means, 
for a given time step, our gradient depends on the loss computed at that time step 
as well as the loss computed for all the time steps after it. 
~\\
~\\
In practice this isn't too difficult. The time step $h_{t + 1}$ will pass back some 
gradient to time step $h_t$. Since this gradient is passed back from every time 
step ahead of $h_t$, it embeds all the gradient with respect to all the loss functions 
after $h_t$ (so excluding the loss function calculated on time step $h_t$. Thus, 
we simply add the gradient passed back from $h_{t+1}$ and the gradient with respect 
to the loss function computed at $h_t$ and pass this value to our 
\emph{rnn-step-backward} function. And that's it! 
~\\
~\\
Again, the details here are pretty simply once you understand the high level 
concept so I won't explain the code. We're just backpropagating like normal 
with a couple extra steps added in between. 

\section{RNN for Image Captioning}

Initially we implement some functions that we need to do our image captioning. Namely, 
we look at MobileNet v2 architecture, create vectors for our word embeddings, and 
look at a temporal affine layer. The first interesting oneto me was the temporal softmax 
loss, which computes the loss at every time step for all the vocabulary and sums them. 





\end{document}