\documentclass[12pt]{article}
\usepackage{lingmacros}
\usepackage{tree-dvips}
% hyper links
\usepackage{hyperref}
\usepackage[utf8]{inputenc}
\usepackage{amsmath}
\usepackage{amsfonts}
\usepackage{bbm}
% Formatting quotes properly
\usepackage[english]{babel}
\usepackage[autostyle, english = american]{csquotes}
\MakeOuterQuote{"}


\begin{document}
\noindent Author: Benjamin Smidt

\noindent Created: September 19, 2022

\noindent Last Updated: September 19, 2022
\begin{center}
\section*{Assignment 2: Full Connected Networks}
\end{center}

\paragraph{} \emph{Note to reader.} 

This is my work for assignment three of Michigan's course
EECS 498: Deep Learning for Computer Vision. This document is meant to be used as a reference, 
explanation, and resource for the assignment, not necessarily a comprehensive overview
of Neural Networks. Furthermore, this document is well-informed and thoroughly researched but
may not be perfect. If there's a typo or a correction needs to be made, feel free to 
email me at benjamin.smidt@utexas.edu so I can fix it. Thank you! I hope you find this 
document helpful.

\tableofcontents{}

\section{Functions}

\subsection{Linear.Forward}
The \emph{Forward} function, implemented in the \emph{Linear} class is simple. We aren't doing 
anything different from the matrix multiplication and bias computed in the two layer
neural network we built last assignment. See \href{https://github.com/bensmidt/EECS-498-DL-Computer-Vision/blob/main/A2/A2-Two-Layer-NN.pdf}
{A2: Two Layer Neural Net} for an in depth explanation of how this matrix multiplication was 
computed. 

\subsection{Linear.Backward}
The \emph{Backward} function, implemented in the \emph{Linear} class certainly isn't
as simply as the forward function. However, I explained how to compute the gradient of 
a matrix multiplication for $x$, $w$, and $b$ in great length in the last assignment 
\href{https://github.com/bensmidt/EECS-498-DL-Computer-Vision/blob/main/A2/A2-Two-Layer-NN.pdf}
{A2: Two Layer Neural Net}.

\subsection{ReLU.Forward}
Just look up the documentation for torch.clamp() if you're confused. It's just setting
every value less than zero equal to zero and leaving positive values alone. 

\subsection{ReLU.Backward}
This is getting repetitive. Again, see A2: Two Layer Neural Net for information on ReLU. 
I explain, in depth, both the forward and backward pass of ReLU including the gradient 
and how to compute it. 

\subsection{Linear-ReLU.forward and Linear-ReLU.backward}
These functions are just utilizing the functions we just created. Although, it's worth 
noting how beautiful the modularity of this code truly is. With no extra work we were 
able to easily define a common function. Obviously, I will not be explaining object oriented 
programming (OOP). That is literally a course in and of itself. However, I did just want to 
stop for a second and marvel at the beauty that OOP has produced here. 

\subsection{Softmax and SVM}
Since we spent a LOT of time deriving, explaining, and programming the SVM and Softmax
algorithms in assigment 2, they were nice enough to gift us this implementation. To be clear, 
the following code is NOT my own. It is freely available from the 
\href{https://web.eecs.umich.edu/~justincj/teaching/eecs498/WI2022/assignment3.html} 
{EECS 498 Course Website} in assignment 3. If you'd like to see my implementation (which is almost
certainly slightly worse), you can check out 
\href{https://github.com/bensmidt/EECS-498-DL-Computer-Vision/blob/main/A2/A2-Softmax.pdf}{A2: Softmax}
or \href{https://github.com/bensmidt/EECS-498-DL-Computer-Vision/blob/main/A2/A2-SVM.pdf}{A2: Multiclass SVM}.

\paragraph{}
\textbf{SVM-Loss-Gradient}
\begin{verbatim}
def svm_loss(x, y):
    """
    Computes the loss and gradient using for multiclass SVM classification.
    Inputs:
    - x: Input data, of shape (N, C) where x[i, j] is the score for the jth
      class for the ith input.
    - y: Vector of labels, of shape (N,) where y[i] is the label for x[i] and
      0 <= y[i] < C
    Returns a tuple of:
    - loss: Scalar giving the loss
    - dx: Gradient of the loss with respect to x
    """
    N = x.shape[0]
    correct_class_scores = x[torch.arange(N), y]
    margins = (x - correct_class_scores[:, None] + 1.0).clamp(min=0)
    margins[torch.arange(N), y] = 0.
    loss = margins.sum() / N
    num_pos = (margins > 0).sum(dim=1)
    dx = torch.zeros_like(x)
    dx[margins > 0] = 1.
    dx[torch.arange(N), y] -= num_pos.to(dx.dtype)
    dx /= N
    return loss, dx
\end{verbatim}

Let's quickly review how this code computes SVM loss. 

\begin{equation}
    L_{i} = \sum_{j \neq y_{i}}^C max(0, x_{j} - x_{y_{i}} + \Delta)
\end{equation}

First, we get the correct class scores. Then we compute the margins by 
subtracting each value from the correct score in it's example, 
adding $\Delta$ (=1), and finally taking the max of that value and zero. 
Remember, this has the interpretation of SVM wanting the difference between 
the correct class score and every other score to be $\geq \Delta$. Lastly, 
we set all the correct class scores equal to zero and average the losses 
over each example. 

The derivatives of SVM with respect to $x_j$ and $x_{y_i}$ are as follows. 
\begin{equation}
    \nabla_{x_{y_i}} L_i = -\sum_{j \neq y_i}^C \mathbbm{1}
        (x_j - x_{y_i} + \Delta > 0)
\end{equation}

\begin{equation}
    \nabla_{x_j} L_i = \sum_{j \neq y_i}^C \mathbbm{1}
        (x_j - x_{y_i} + \Delta > 0)
\end{equation}

This is somewhat remiscient of the ReLU function since we have to deal with the 
$max()$ function. Regarding code though, for each example we simply sum the number of terms 
with a margin greater than zero. Then we set the gradient value at a given example N
equal to this value, giving us the gradient with respect to $x_{y_i}$. For the gradient
with respect to $x_j$, we simply set all the incorrect scores index values to 1 if their
margin exceeds 0 and (leave it at) zero otherwise. Finally we divide all the values by 
N since we average the values over N examples (if we summed them then we wouldn't divide
all the values by N). 

\paragraph{}
\textbf{Softmax-Loss-Gradient}
\begin{verbatim}
def softmax_loss(x, y):
    """
    Computes the loss and gradient for softmax classification.
    Inputs:
    - x: Input data, of shape (N, C) where x[i, j] is the score for
      the jth class for the ith input.
    - y: Vector of labels, of shape (N,) where y[i] is the label
      for x[i] and 0 <= y[i] < C
    Returns a tuple of:
    - loss: Scalar giving the loss
    - dx: Gradient of the loss with respect to x
    """
    shifted_logits = x - x.max(dim=1, keepdim=True).values
    Z = shifted_logits.exp().sum(dim=1, keepdim=True)
    log_probs = shifted_logits - Z.log()
    probs = log_probs.exp()
    N = x.shape[0]
    loss = (-1.0 / N) * log_probs[torch.arange(N), y].sum()
    dx = probs.clone()
    dx[torch.arange(N), y] -= 1
    dx /= N
    return loss, dx
\end{verbatim}

For softmax, our loss is: 

\begin{equation}
    L_{i} = -log(\frac{e^{x_{y_{i}}}}{\sum_{j=1}^C e^{x_j}}) 
    = -x_{y_{i}} + log(\sum_{j=1}^C e^{x_{j}})
\end{equation}

In the code, we first normalize the everything to improve numerical stability. 
We can shift everything by any constant K and the loss won't change. 

\begin{equation}
    \frac{K e^{x_{y_{i}}}}{K \sum_{j=1}^C e^{x_j}} = 
    \frac{e^{x_{y_{i}} + logK}}{\sum_{j=1}^C e^{x_j + log K}} 
\end{equation}

For simplicity, we often choose $log K$ to be the negative of the max value in each example 
$e^x$. This prevents any single $e^{x_j}$ value from becoming too large
and subsequently producing a NaN or infinity. 

Anyways, we compute the loss function by adding the correct class score values
to the log of the sum of the scores in each example. Then we multiply by negative 1
and average over N examples. Personall, I like to distribut the negative 1 instead 
of multiplying at the end as seen in my loss function equation above, but it's all
the same at the end of the day. 

Lasly, here are the derivatives. 

\begin{equation}
    \frac{\partial L_{i}}{\partial x_{y_i}} = \frac{e^{x_{y_i}}}
    {\sum_{j=1}^C e^{x_{j}}} - 1
\end{equation}

\begin{equation}
    \frac{\partial L_{i}}{\partial x_j} = \frac{e^{x_{j}}}
    {\sum_{j=1}^C e^{x_j}}
\end{equation}

To compute the gradient we simply reuse the \emph{probs} variable which stores
all the values of both gradients we need. We simply alter the correct scores
indices by subtracting 1 and finish by dividing by N since we average the loss 
over N examples. 

\subsection{TwoLayerNet Class}

\section{References}

\end{document}